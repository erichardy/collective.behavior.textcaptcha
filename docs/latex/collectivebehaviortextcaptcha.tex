% Generated by Sphinx.
\def\sphinxdocclass{report}
\documentclass[letterpaper,10pt,french]{sphinxmanual}
\usepackage[utf8]{inputenc}
\DeclareUnicodeCharacter{00A0}{\nobreakspace}
\usepackage{cmap}
\usepackage[T1]{fontenc}
\usepackage{babel}
\usepackage{times}
\usepackage[Sonny]{fncychap}
\usepackage{longtable}
\usepackage{sphinx}
\usepackage{multirow}
\usepackage{eqparbox}
\usepackage{amsfonts}

\addto\captionsfrench{\renewcommand{\figurename}{Fig. }}
\addto\captionsfrench{\renewcommand{\tablename}{Tableau }}
\SetupFloatingEnvironment{literal-block}{name=Code source }



\title{collective.behavior.textcaptcha Documentation}
\date{29 January 2016}
\release{1.0}
\author{Eric Hardy}
\newcommand{\sphinxlogo}{}
\renewcommand{\releasename}{Version}
\setcounter{tocdepth}{1}
\makeindex

\makeatletter
\def\PYG@reset{\let\PYG@it=\relax \let\PYG@bf=\relax%
    \let\PYG@ul=\relax \let\PYG@tc=\relax%
    \let\PYG@bc=\relax \let\PYG@ff=\relax}
\def\PYG@tok#1{\csname PYG@tok@#1\endcsname}
\def\PYG@toks#1+{\ifx\relax#1\empty\else%
    \PYG@tok{#1}\expandafter\PYG@toks\fi}
\def\PYG@do#1{\PYG@bc{\PYG@tc{\PYG@ul{%
    \PYG@it{\PYG@bf{\PYG@ff{#1}}}}}}}
\def\PYG#1#2{\PYG@reset\PYG@toks#1+\relax+\PYG@do{#2}}

\expandafter\def\csname PYG@tok@gd\endcsname{\def\PYG@tc##1{\textcolor[rgb]{0.63,0.00,0.00}{##1}}}
\expandafter\def\csname PYG@tok@gu\endcsname{\let\PYG@bf=\textbf\def\PYG@tc##1{\textcolor[rgb]{0.50,0.00,0.50}{##1}}}
\expandafter\def\csname PYG@tok@gt\endcsname{\def\PYG@tc##1{\textcolor[rgb]{0.00,0.27,0.87}{##1}}}
\expandafter\def\csname PYG@tok@gs\endcsname{\let\PYG@bf=\textbf}
\expandafter\def\csname PYG@tok@gr\endcsname{\def\PYG@tc##1{\textcolor[rgb]{1.00,0.00,0.00}{##1}}}
\expandafter\def\csname PYG@tok@cm\endcsname{\let\PYG@it=\textit\def\PYG@tc##1{\textcolor[rgb]{0.25,0.50,0.56}{##1}}}
\expandafter\def\csname PYG@tok@vg\endcsname{\def\PYG@tc##1{\textcolor[rgb]{0.73,0.38,0.84}{##1}}}
\expandafter\def\csname PYG@tok@m\endcsname{\def\PYG@tc##1{\textcolor[rgb]{0.13,0.50,0.31}{##1}}}
\expandafter\def\csname PYG@tok@mh\endcsname{\def\PYG@tc##1{\textcolor[rgb]{0.13,0.50,0.31}{##1}}}
\expandafter\def\csname PYG@tok@cs\endcsname{\def\PYG@tc##1{\textcolor[rgb]{0.25,0.50,0.56}{##1}}\def\PYG@bc##1{\setlength{\fboxsep}{0pt}\colorbox[rgb]{1.00,0.94,0.94}{\strut ##1}}}
\expandafter\def\csname PYG@tok@ge\endcsname{\let\PYG@it=\textit}
\expandafter\def\csname PYG@tok@vc\endcsname{\def\PYG@tc##1{\textcolor[rgb]{0.73,0.38,0.84}{##1}}}
\expandafter\def\csname PYG@tok@il\endcsname{\def\PYG@tc##1{\textcolor[rgb]{0.13,0.50,0.31}{##1}}}
\expandafter\def\csname PYG@tok@go\endcsname{\def\PYG@tc##1{\textcolor[rgb]{0.20,0.20,0.20}{##1}}}
\expandafter\def\csname PYG@tok@cp\endcsname{\def\PYG@tc##1{\textcolor[rgb]{0.00,0.44,0.13}{##1}}}
\expandafter\def\csname PYG@tok@gi\endcsname{\def\PYG@tc##1{\textcolor[rgb]{0.00,0.63,0.00}{##1}}}
\expandafter\def\csname PYG@tok@gh\endcsname{\let\PYG@bf=\textbf\def\PYG@tc##1{\textcolor[rgb]{0.00,0.00,0.50}{##1}}}
\expandafter\def\csname PYG@tok@ni\endcsname{\let\PYG@bf=\textbf\def\PYG@tc##1{\textcolor[rgb]{0.84,0.33,0.22}{##1}}}
\expandafter\def\csname PYG@tok@nl\endcsname{\let\PYG@bf=\textbf\def\PYG@tc##1{\textcolor[rgb]{0.00,0.13,0.44}{##1}}}
\expandafter\def\csname PYG@tok@nn\endcsname{\let\PYG@bf=\textbf\def\PYG@tc##1{\textcolor[rgb]{0.05,0.52,0.71}{##1}}}
\expandafter\def\csname PYG@tok@no\endcsname{\def\PYG@tc##1{\textcolor[rgb]{0.38,0.68,0.84}{##1}}}
\expandafter\def\csname PYG@tok@na\endcsname{\def\PYG@tc##1{\textcolor[rgb]{0.25,0.44,0.63}{##1}}}
\expandafter\def\csname PYG@tok@nb\endcsname{\def\PYG@tc##1{\textcolor[rgb]{0.00,0.44,0.13}{##1}}}
\expandafter\def\csname PYG@tok@nc\endcsname{\let\PYG@bf=\textbf\def\PYG@tc##1{\textcolor[rgb]{0.05,0.52,0.71}{##1}}}
\expandafter\def\csname PYG@tok@nd\endcsname{\let\PYG@bf=\textbf\def\PYG@tc##1{\textcolor[rgb]{0.33,0.33,0.33}{##1}}}
\expandafter\def\csname PYG@tok@ne\endcsname{\def\PYG@tc##1{\textcolor[rgb]{0.00,0.44,0.13}{##1}}}
\expandafter\def\csname PYG@tok@nf\endcsname{\def\PYG@tc##1{\textcolor[rgb]{0.02,0.16,0.49}{##1}}}
\expandafter\def\csname PYG@tok@si\endcsname{\let\PYG@it=\textit\def\PYG@tc##1{\textcolor[rgb]{0.44,0.63,0.82}{##1}}}
\expandafter\def\csname PYG@tok@s2\endcsname{\def\PYG@tc##1{\textcolor[rgb]{0.25,0.44,0.63}{##1}}}
\expandafter\def\csname PYG@tok@vi\endcsname{\def\PYG@tc##1{\textcolor[rgb]{0.73,0.38,0.84}{##1}}}
\expandafter\def\csname PYG@tok@nt\endcsname{\let\PYG@bf=\textbf\def\PYG@tc##1{\textcolor[rgb]{0.02,0.16,0.45}{##1}}}
\expandafter\def\csname PYG@tok@nv\endcsname{\def\PYG@tc##1{\textcolor[rgb]{0.73,0.38,0.84}{##1}}}
\expandafter\def\csname PYG@tok@s1\endcsname{\def\PYG@tc##1{\textcolor[rgb]{0.25,0.44,0.63}{##1}}}
\expandafter\def\csname PYG@tok@gp\endcsname{\let\PYG@bf=\textbf\def\PYG@tc##1{\textcolor[rgb]{0.78,0.36,0.04}{##1}}}
\expandafter\def\csname PYG@tok@sh\endcsname{\def\PYG@tc##1{\textcolor[rgb]{0.25,0.44,0.63}{##1}}}
\expandafter\def\csname PYG@tok@ow\endcsname{\let\PYG@bf=\textbf\def\PYG@tc##1{\textcolor[rgb]{0.00,0.44,0.13}{##1}}}
\expandafter\def\csname PYG@tok@sx\endcsname{\def\PYG@tc##1{\textcolor[rgb]{0.78,0.36,0.04}{##1}}}
\expandafter\def\csname PYG@tok@bp\endcsname{\def\PYG@tc##1{\textcolor[rgb]{0.00,0.44,0.13}{##1}}}
\expandafter\def\csname PYG@tok@c1\endcsname{\let\PYG@it=\textit\def\PYG@tc##1{\textcolor[rgb]{0.25,0.50,0.56}{##1}}}
\expandafter\def\csname PYG@tok@kc\endcsname{\let\PYG@bf=\textbf\def\PYG@tc##1{\textcolor[rgb]{0.00,0.44,0.13}{##1}}}
\expandafter\def\csname PYG@tok@c\endcsname{\let\PYG@it=\textit\def\PYG@tc##1{\textcolor[rgb]{0.25,0.50,0.56}{##1}}}
\expandafter\def\csname PYG@tok@mf\endcsname{\def\PYG@tc##1{\textcolor[rgb]{0.13,0.50,0.31}{##1}}}
\expandafter\def\csname PYG@tok@err\endcsname{\def\PYG@bc##1{\setlength{\fboxsep}{0pt}\fcolorbox[rgb]{1.00,0.00,0.00}{1,1,1}{\strut ##1}}}
\expandafter\def\csname PYG@tok@mb\endcsname{\def\PYG@tc##1{\textcolor[rgb]{0.13,0.50,0.31}{##1}}}
\expandafter\def\csname PYG@tok@ss\endcsname{\def\PYG@tc##1{\textcolor[rgb]{0.32,0.47,0.09}{##1}}}
\expandafter\def\csname PYG@tok@sr\endcsname{\def\PYG@tc##1{\textcolor[rgb]{0.14,0.33,0.53}{##1}}}
\expandafter\def\csname PYG@tok@mo\endcsname{\def\PYG@tc##1{\textcolor[rgb]{0.13,0.50,0.31}{##1}}}
\expandafter\def\csname PYG@tok@kd\endcsname{\let\PYG@bf=\textbf\def\PYG@tc##1{\textcolor[rgb]{0.00,0.44,0.13}{##1}}}
\expandafter\def\csname PYG@tok@mi\endcsname{\def\PYG@tc##1{\textcolor[rgb]{0.13,0.50,0.31}{##1}}}
\expandafter\def\csname PYG@tok@kn\endcsname{\let\PYG@bf=\textbf\def\PYG@tc##1{\textcolor[rgb]{0.00,0.44,0.13}{##1}}}
\expandafter\def\csname PYG@tok@o\endcsname{\def\PYG@tc##1{\textcolor[rgb]{0.40,0.40,0.40}{##1}}}
\expandafter\def\csname PYG@tok@kr\endcsname{\let\PYG@bf=\textbf\def\PYG@tc##1{\textcolor[rgb]{0.00,0.44,0.13}{##1}}}
\expandafter\def\csname PYG@tok@s\endcsname{\def\PYG@tc##1{\textcolor[rgb]{0.25,0.44,0.63}{##1}}}
\expandafter\def\csname PYG@tok@kp\endcsname{\def\PYG@tc##1{\textcolor[rgb]{0.00,0.44,0.13}{##1}}}
\expandafter\def\csname PYG@tok@w\endcsname{\def\PYG@tc##1{\textcolor[rgb]{0.73,0.73,0.73}{##1}}}
\expandafter\def\csname PYG@tok@kt\endcsname{\def\PYG@tc##1{\textcolor[rgb]{0.56,0.13,0.00}{##1}}}
\expandafter\def\csname PYG@tok@sc\endcsname{\def\PYG@tc##1{\textcolor[rgb]{0.25,0.44,0.63}{##1}}}
\expandafter\def\csname PYG@tok@sb\endcsname{\def\PYG@tc##1{\textcolor[rgb]{0.25,0.44,0.63}{##1}}}
\expandafter\def\csname PYG@tok@k\endcsname{\let\PYG@bf=\textbf\def\PYG@tc##1{\textcolor[rgb]{0.00,0.44,0.13}{##1}}}
\expandafter\def\csname PYG@tok@se\endcsname{\let\PYG@bf=\textbf\def\PYG@tc##1{\textcolor[rgb]{0.25,0.44,0.63}{##1}}}
\expandafter\def\csname PYG@tok@sd\endcsname{\let\PYG@it=\textit\def\PYG@tc##1{\textcolor[rgb]{0.25,0.44,0.63}{##1}}}

\def\PYGZbs{\char`\\}
\def\PYGZus{\char`\_}
\def\PYGZob{\char`\{}
\def\PYGZcb{\char`\}}
\def\PYGZca{\char`\^}
\def\PYGZam{\char`\&}
\def\PYGZlt{\char`\<}
\def\PYGZgt{\char`\>}
\def\PYGZsh{\char`\#}
\def\PYGZpc{\char`\%}
\def\PYGZdl{\char`\$}
\def\PYGZhy{\char`\-}
\def\PYGZsq{\char`\'}
\def\PYGZdq{\char`\"}
\def\PYGZti{\char`\~}
% for compatibility with earlier versions
\def\PYGZat{@}
\def\PYGZlb{[}
\def\PYGZrb{]}
\makeatother

\renewcommand\PYGZsq{\textquotesingle}

\begin{document}

\maketitle
\tableofcontents
\phantomsection\label{index::doc}


Documentation de \code{collective.behavior.textcaptcha} dévéloppé à l'\href{http://www-iuem.univ-brest.fr}{IUEM}.

Voir les recommandations pour la documentation à \href{http://docs.plone.org/about/documentation\_styleguide\_addons.html}{DocPlone}

Voir aussi Sphinx : \href{http://sphinx-doc.org/}{Sphinx}


\chapter{Installation}
\label{index:installation}\label{index:documentation-de-collective-behavior-textcaptcha}
Ajouter \emph{collective.behavior.textcaptcha} à la liste definie par la variable \code{eggs} dans la
section \code{{[}instance{]}} du fichier \emph{buildout.cfg}

et la source dans la section \code{{[}sources{]}}:

\begin{Verbatim}[commandchars=\\\{\}]
collective.behavior.textcaptcha = git gitiuem:collective.behavior.textcaptcha.git
\end{Verbatim}


\chapter{Usage}
\label{index:usage}
Le module collective.behavior.textcaptcha a pour fonction de remplacer les \href{https://fr.wikipedia.org/wiki/CAPTCHA}{captchas}
ordinaires qui font généralement appel à des images dont on doit recopier le textes.

Le principe est simple :
\begin{itemize}
\item {} 
on crée une liste de mots qui contiennent des caractères comme des espaces, le tiret (\code{-})
ou \emph{underscore} (\code{\_}). Cette liste mise en place par le \emph{Control Panel} de l'instance plone

\item {} 
on utilise ce \emph{behavior} comme champ complémentaire d'un type de contenu \emph{dexterity}

\end{itemize}

Dans le formulaire, qui est créé par \emph{dexterity}, deux champs texte apparaissent :
\begin{itemize}
\item {} 
le champ contenant le captcha à recopier. Par exemple : \code{b re-ta\_gn e}. Ce champ
est \emph{Read Only}

\item {} 
un champ texte où l'utilisateur doit saisir \code{bretagne}. Si la saisie n'est
pas correcte le formulaire ne peut pas être envoyé.

\end{itemize}

Ce module est utilisé dans les applications suivantes :
\begin{itemize}
\item {} 
\code{ueb.thesesenbretagne}

\item {} 
\code{iuem.proposal}

\end{itemize}

Pour ces deux applications, des contenus sont créés par des utilisateurs anonymes,
et les formulaires doivent être \emph{protégés} contre les robots.

Le mécanisme de captcha utilisé ici est très simple... peut-être même simpliste,
mais est léger, simple à installer et à configurer.


\chapter{Toute la documentation}
\label{index:toute-la-documentation}

\section{Le behavior collective.behavior.textcaptcha}
\label{behavior:le-behavior-collective-behavior-textcaptcha}\label{behavior::doc}

\subsection{Configuration zcml}
\label{behavior:configuration-zcml}
Le point de départ, le fichier \code{configure.zcml} qui contient d'une part la
déclaration de la vue du control panel:

\begin{Verbatim}[commandchars=\\\{\}]
\PYGZlt{}browser:page
  name=\PYGZdq{}textcaptcha\PYGZhy{}settings\PYGZdq{}
  for=\PYGZdq{}Products.CMFPlone.interfaces.IPloneSiteRoot\PYGZdq{}
  class=\PYGZdq{}.controlpanel.TextCaptchaSettingsFormSettingsControlPanel\PYGZdq{}
  permission=\PYGZdq{}cmf.ManagePortal\PYGZdq{}
/\PYGZgt{}
\end{Verbatim}

d'autre part l'appel à un autre fichier zcml:

\begin{Verbatim}[commandchars=\\\{\}]
\PYGZlt{}include file=\PYGZdq{}behaviors.zcml\PYGZdq{} /\PYGZgt{}
\end{Verbatim}

qui contient la déclaration du behavior lui-même:

\begin{Verbatim}[commandchars=\\\{\}]
\PYGZlt{}plone:behavior
    title=\PYGZdq{}text captcha\PYGZdq{}
    description=\PYGZdq{}Provides a new field: a captcha text field\PYGZdq{}
    provides=\PYGZdq{}.behaviors.textcaptcha.ITextCaptcha\PYGZdq{}
    factory=\PYGZdq{}.behaviors.textcaptcha.textCaptcha\PYGZdq{}
    marker=\PYGZdq{}.behaviors.textcaptcha.ITextCaptchaMarker\PYGZdq{}
    /\PYGZgt{}
\end{Verbatim}


\subsection{Le code}
\label{behavior:le-code}\label{behavior:module-collective.behavior.textcaptcha.behaviors.textcaptcha}\index{collective.behavior.textcaptcha.behaviors.textcaptcha (module)}\index{CaptchaNotValid}

\begin{fulllineitems}
\phantomsection\label{behavior:collective.behavior.textcaptcha.behaviors.textcaptcha.CaptchaNotValid}\pysigline{\strong{exception }\code{collective.behavior.textcaptcha.behaviors.textcaptcha.}\bfcode{CaptchaNotValid}}
Please, enter the good value here !
\index{justForDocumentation() (méthode collective.behavior.textcaptcha.behaviors.textcaptcha.CaptchaNotValid)}

\begin{fulllineitems}
\phantomsection\label{behavior:collective.behavior.textcaptcha.behaviors.textcaptcha.CaptchaNotValid.justForDocumentation}\pysiglinewithargsret{\bfcode{justForDocumentation}}{}{}
Exception levée si le captcha n'est pas valide. Le message affiché
si le texte saisi par l'utilisateur n'est pas valide est contenu dans
\code{\_\_doc\_\_}.

\end{fulllineitems}


\end{fulllineitems}

\index{SampleValidator (classe dans collective.behavior.textcaptcha.behaviors.textcaptcha)}

\begin{fulllineitems}
\phantomsection\label{behavior:collective.behavior.textcaptcha.behaviors.textcaptcha.SampleValidator}\pysiglinewithargsret{\strong{class }\code{collective.behavior.textcaptcha.behaviors.textcaptcha.}\bfcode{SampleValidator}}{\emph{context}, \emph{request}, \emph{view}, \emph{field}, \emph{widget}}{}~\index{validate() (méthode collective.behavior.textcaptcha.behaviors.textcaptcha.SampleValidator)}

\begin{fulllineitems}
\phantomsection\label{behavior:collective.behavior.textcaptcha.behaviors.textcaptcha.SampleValidator.validate}\pysiglinewithargsret{\bfcode{validate}}{\emph{value}}{}~\begin{quote}\begin{description}
\item[{Paramètres}] \leavevmode
\textbf{\texttt{value}} (\href{https://docs.python.org/library/functions.html\#str}{\emph{\texttt{str}}}) -- la chaine contenue dans le champ \code{captcha\_input}

\item[{Retourne}] \leavevmode
True ou lève l'exception \code{CaptchaNotValid}

\end{description}\end{quote}

\end{fulllineitems}


\end{fulllineitems}

\index{randomCaptcha() (dans le module collective.behavior.textcaptcha.behaviors.textcaptcha)}

\begin{fulllineitems}
\phantomsection\label{behavior:collective.behavior.textcaptcha.behaviors.textcaptcha.randomCaptcha}\pysiglinewithargsret{\code{collective.behavior.textcaptcha.behaviors.textcaptcha.}\bfcode{randomCaptcha}}{\emph{self}}{}
\end{fulllineitems}

\index{randomCaptchaLabel() (dans le module collective.behavior.textcaptcha.behaviors.textcaptcha)}

\begin{fulllineitems}
\phantomsection\label{behavior:collective.behavior.textcaptcha.behaviors.textcaptcha.randomCaptchaLabel}\pysiglinewithargsret{\code{collective.behavior.textcaptcha.behaviors.textcaptcha.}\bfcode{randomCaptchaLabel}}{}{}~\begin{quote}\begin{description}
\item[{Retourne}] \leavevmode
un des choix possibles de captcha mis dans le control panel

\end{description}\end{quote}

\end{fulllineitems}

\index{textCaptcha (classe dans collective.behavior.textcaptcha.behaviors.textcaptcha)}

\begin{fulllineitems}
\phantomsection\label{behavior:collective.behavior.textcaptcha.behaviors.textcaptcha.textCaptcha}\pysiglinewithargsret{\strong{class }\code{collective.behavior.textcaptcha.behaviors.textcaptcha.}\bfcode{textCaptcha}}{\emph{context}}{}
La classe du formulaire lui-même, composé seulement de deux champs
texte, \code{captcha\_value} (read only) et \code{captcha\_input}
\index{description (attribut collective.behavior.textcaptcha.behaviors.textcaptcha.textCaptcha)}

\begin{fulllineitems}
\phantomsection\label{behavior:collective.behavior.textcaptcha.behaviors.textcaptcha.textCaptcha.description}\pysigline{\bfcode{description}\strong{ = u'Provides a new field: a captcha text field'}}
\end{fulllineitems}

\index{ignoreContext (attribut collective.behavior.textcaptcha.behaviors.textcaptcha.textCaptcha)}

\begin{fulllineitems}
\phantomsection\label{behavior:collective.behavior.textcaptcha.behaviors.textcaptcha.textCaptcha.ignoreContext}\pysigline{\bfcode{ignoreContext}\strong{ = True}}
\end{fulllineitems}

\index{title (attribut collective.behavior.textcaptcha.behaviors.textcaptcha.textCaptcha)}

\begin{fulllineitems}
\phantomsection\label{behavior:collective.behavior.textcaptcha.behaviors.textcaptcha.textCaptcha.title}\pysigline{\bfcode{title}\strong{ = u'text captcha'}}
\end{fulllineitems}


\end{fulllineitems}



\renewcommand{\indexname}{Index des modules Python}
\begin{theindex}
\def\bigletter#1{{\Large\sffamily#1}\nopagebreak\vspace{1mm}}
\bigletter{c}
\item {\texttt{collective.behavior.textcaptcha.behaviors.textcaptcha}}, \pageref{behavior:module-collective.behavior.textcaptcha.behaviors.textcaptcha}
\end{theindex}

\renewcommand{\indexname}{Index}
\printindex
\end{document}
